\documentclass{article}
\usepackage{hyperref}
\usepackage[margin=1in]{geometry}

\begin{document}
\title{The Chomsky Hierarchy}
\author{Spencer Hirsch, Thomas Johnson, Tyler Gutowski, Remington Greko}
\date{\today}

\maketitle

\noindent Noam Chomsky is a reowned linguist whose ideas were recognized and applied
by computer scientists to lay foundations of our discipline. I admire him 
for his support of justice.

\begin{itemize}
    \item Read about Noam Chomsky on Wikipedia and summarize his salient 
          contributions to our computing discipline.
    
    \item Outline the Chomsky Hierarchy giving what we now call
        \begin{itemize}
            \item Regular Languages
            \item Context-Free Languages
            \item Context-Sensitive Languages
            \item Unrestricted Languages
        \end{itemize}

    \item Read about Grammars \href{https://en.wikipedia.org/wiki/Grammar}{Wikipedia}. Report on what you find interesting.
\end{itemize}

\section{Chomsky}
(\textbf{???})

\section{Chomsky Hierarchy}  
\textbf{TYLER \& ???}

\subsection{Regular Languages} (\textbf{TYLER})

\subsection{Context-Free Languages} (\textbf{TYLER})

\subsection{Context-Sensitive Languages} (\textbf{???})

\subsection{Unrestricted Languages} (\textbf{???})

\section{Grammars}

\textbf{Spencer}

\medskip

Grammar can be used to describe behavior of groups of speakers, grammar can be used to demonstrate the language of not only an
entire group of individuals but also subsets within that group. It's interesting how grammar changes depending on what sample 
of people you are looking at. Geography has an influence on the way that individuals speak.

With regards to history, it is interesting that the ``earliest known grammar handbook is the \textit{Art of Grammar}'' was written
by an ancient Greek scholar in 90 BC. Throughout history grammar has been based off of the grammar of other languages, each society
basing it's grammar off of an older language, demonstrating that grammar that is used today have evolved from ancient languages.
Grammar has also been a topic of study throughout many time periods including the Middle Ages.

Grammar evolves with time and usage most notabily through the observation and documentation. Sentences that have slight variations in
language can have rather large differences based on context. There is no clear line between syntax and morphology




\end{document}
