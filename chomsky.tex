\documentclass{article}
\usepackage{hyperref}
\usepackage[margin=1in]{geometry}
\usepackage{indentfirst}   % Indents first paragraph. change if u want ig

\begin{document}
\title{The Chomsky Hierarchy}
\author{Spencer Hirsch, Thomas Johnson, Tyler Gutowski, Remington Greko}
\date{\today}

\maketitle

\noindent Noam Chomsky is a reowned linguist whose ideas were recognized and applied
by computer scientists to lay foundations of our discipline. I admire him 
for his support of justice.

\begin{itemize}
    \item Read about Noam Chomsky on Wikipedia and summarize his salient 
          contributions to our computing discipline.
    
    \item Outline the Chomsky Hierarchy giving what we now call
        \begin{itemize}
            \item Regular Languages
            \item Context-Free Languages
            \item Context-Sensitive Languages
            \item Unrestricted Languages
        \end{itemize}

    \item Read about Grammars \href{https://en.wikipedia.org/wiki/Grammar}{Wikipedia}. Report on what you find interesting.
\end{itemize}

\section{Chomsky}
% (\textbf{THOMAS})
\medskip

Avram Noam Chomsky born December 7th 1928 is an American linguist, philosopher, political activist, and many more. 
Dubbed the father of modern linguistics, Chomsky helped establish the field of cognitive science.
\medskip

Chomsky's early works had him studying at the University of Pennsylvania to which post graduation he began studying at Harvard Society of Fellows. 
From here he earned his doctorate in 1955 for his theory on transformational grammar, and  released the book Syntactic Structures in 1957 which helped redevelop the study of language. 
In the years following Chomsky helped create the universal grammar theory, and the famous Chomsky hierarchy. He then went on to help create the minimalist program, and lead a life of activism. 
He has since settled down to teach at the University of Arizona as of 2017.
\medskip

In the context of computer science Chomsky has a few very notable works. 
His book Syntactic Structures contains his theory on transformational-generative grammar (originally published in his 1955 dissertation), and the Chomsky hierarchy.
\medskip

Syntactic Structures covers the formality of syntax, and the symbols and rules comprising it. He does this via something he refers to as phase structures. 
He then applies his new rules called transformations. Without going too in depth these transformations through mathematical notation can generate all grammatical sentences of a language. 
This is the basis of formal languages, and how they may be constructed via grammars. 
\medskip

The Chomsky Hierarchy is a way of grouping all logical possible phrase-structure grammars into nested subsets with these grammars becoming more complex as we move up. 
This hierarchy contributed greatly to the study of formal language theory, programming language theory, compiler construction, and automata theory.


\section{Chomsky Hierarchy}  
\textbf{TYLER \& ???}

\subsection{Regular Languages} (\textbf{TYLER})

\subsection{Context-Free Languages} (\textbf{TYLER})

\subsection{Context-Sensitive Languages} (\textbf{???})

\subsection{Unrestricted Languages} (\textbf{???})

\section{Grammars}

\medskip

Grammar can be used to describe behavior of groups of speakers, grammar can be used to demonstrate the language of not only an
entire group of individuals but also subsets within that group. It's interesting how grammar changes depending on what sample 
of people you are looking at. Geography has an influence on the way that individuals speak.

With regards to history, it is interesting that the ``earliest known grammar handbook is the \textit{Art of Grammar}'' was written
by an ancient Greek scholar in 90 BC. Throughout history grammar has been based off of the grammar of other languages, each society
basing it's grammar off of an older language, demonstrating that grammar that is used today have evolved from ancient languages.
Grammar has also been a topic of study throughout many time periods including the Middle Ages.

Grammar evolves with time and usage most notabily through the observation and documentation. Sentences that have slight variations in
language can have rather large differences based on context. It is interesting to think that grammar has defined rules however, the 
evolution of language changes the definition of such rules. Demonstrating that grammar rules are dicated by those who break grammar 
rules. 

There is a difference in grammar in an academic setting and in the real world, vernacular dialects and standard language respectively.





\end{document}
