\documentclass{article}
\usepackage{hyperref}
\usepackage[margin=1in]{geometry}
\usepackage{indentfirst}
\usepackage{setspace} 
\doublespacing


\begin{document}
\title{Regular Languages and Finite State Machines}
\author{Remington Greko, Tyler Gutowski, Spencer Hirsch, Thomas Johnson}
\date{\today}

\maketitle

\begin{enumerate}
    \item Read the \href{https://en.wikipedia.org/wiki/Regular_grammar}{Wikipedia} article on regular grammars. Summarize
            the saliet points.

    \medskip

    % Answer here

    \medskip

    \item What is a Deterministic Finite Accepter (DFA)?
    
    \medskip

    % Answer here
    \-\hspace{0.5cm} A Deterministic Finite Accepter is a machine that can process an input
    string from left to right. A Finite Accepter is deterministic when there is only one thing
    that it can do for an input symbol. The DFA will either accept or reject the string. Once
    a String is accepted once, it will always be accepted. 

    \-\hspace{0.5cm} A DFA can only read left to right, just as traditional in the English
    language. The DFA can only see one specific element of a string at a time, it cannot go 
    backwards, nor skip ahead. A DFA also has a specific number of internal states, each
    different based on its current situation, such as when beginning a string. 

    \-\hspace{0.5cm} The example from the book is a great example of a Deterministic Finite Accepter,
    
    \[M = (Q, \Sigma, \delta, q_0, F)\]

    where

    \smallskip

    \-\hspace{0.5cm} \textit{Q} is a finite set of \textbf{internal states}, \\
    \-\hspace{0.5cm} $\Sigma$ is a finite set of symbols called the \textbf{input alphabet}, \\
    \-\hspace{0.5cm} $\delta$: \textit{Q} $\times$ $\Sigma$ $\rightarrow$ \textit{Q} is a total function called the 
    \textbf{transition function}, \\
    \-\hspace{0.5cm} \textit{$q_0$ $\in$ Q} is the \textbf{initial state}, \\
    \-\hspace{0.5cm} \textit{F \ $\subseteq$ Q} is a set of \textbf{final states}.



    \medskip

    \item What is a Non-Deterministic Finite Accepter (NFA)?
    
    \medskip

    % Answer here

    \medskip

    \item Explain why the languages accepted by DFAs and NFAs are the equivalent.
    
    \medskip

    % Answer here
    \-\hspace{0.5cm} Any language accepted by a DFA can also be accepted by an NFA,
     and vice versa. We can prove this through the use of the "subset construction
     algorithm." The core principle of this algorithm is the DFA simulates the NFA
     by keeping track of the every possible state. Each state of the DFA corresponds
     to a subset of the sets of the NFA.

     \-\hspace{0.5cm} As long as the languages are equivalent then both a DFA and an 
     NFA can be used when using the language. Although a DFA can only consist of a 
     fixed number of processes, those processes are a subset of the NFA processes,
     therefore languages accepted by both the DFA and the NFA are equivalent.

    \medskip

    \item Give a recursive definition of \textit{regular expression} over
            an alaphabet $\Sigma$.

    \medskip

    % Answer here

    \medskip

    \item Confirm you know how to use operating system commands to find
            regular expressions in a file.

    \medskip

    % Answer here

    \medskip
\end{enumerate}

\pagebreak

\begin{center}
    \begin{tabular}{|p{3cm}|p{6cm}|}
        \hline
        \textbf{Name} & \textbf{Section} \\
        \hline
        Remington Greko &  \\
        \hline
        Tyler Gutowski &  \\
        \hline
        Spencer Hirsch &  Question 2 and 4\\
        \hline
        Thomas Johnson &  \\
        \hline
    \end{tabular}
\end{center}

\end{document}
